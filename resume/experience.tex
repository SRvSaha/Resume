\cvsection{Experience}
\begin{cventries}
    \cventry
    {Guide : Prof. (Dr.) Varun Dutt, IIT Mandi}
    {Data Science Intern, IIT Mandi, Himachal Pradesh}
    {IIT Mandi}
    {\textbf{Dec. 2016 - Jan. 2017}}
    {
      \begin{cvitems}
        \item\textbf {Machine Learning and Data Mining project using Big Data in Health-Care}
        \item{\textbf{Field} : Data Mining, Big Data, Machine Learning}
        \item{Developed a predictive model which could identify from EMR Datasets which patient is likely to buy which medicine using Machine Learning Techniques.}
        \item{Developed a predictive model for determination of Frequent/Infrequent buyer given the attributes of the patient.}
        \item{Case study and analyses of various databases like MongoDB, Cassandra, HBase, kdb+ to understand why kdb+ is better in handling realtime big data.}
        \item{Reduced the time required for training the system by using Weka-Parallel, a parallel computing architecture.}
        \item{Analyses of handling Big Data in Hadoop vs Spark for Machine Learning}
        \item{Built majority voted ensemble for binary-class and ternary-class classification task from scratch.}
        \item{Tuned the performance of Decision Tree ML Algorithm by hyperparameter tuning using GridSearch Algorithm.}
        \item {\textbf{Tools} : Python, Excel, Apache Hadoop, Weka, Weka-Parallel, kdb+}
      \end{cvitems}
    }
     \cventry
    {Guide : Dr. Partha Pakray, HoD CSE, NIT Mizoram, Mr. Sandeeep Dash, NIT Mizoram}
    {Paraphrase Detection in India Languages, FIRE-ISI 2016}
    {NIT Mizoram, Aizawl}
    {\textbf{Jul. 2016 - Sept. 2016}}
    {
      \begin{cvitems}
      \item{\textbf{Field} : Machine Learning, Textual Semantic Similarity}
        \item {Our system  NLP-NITMZ is based on three features: Unigram Matching Ratio, Levenshtein Ratio and Cosine Similarity using Vector Space Model.}
        \item {Built two classifiers which can tag paraphrases, non-paraphrases and semi-phrases in Indian Languages,
namely Hindi, Malayalam, Punjabi and Tamil. Our classifiers are voted ensambles built on the top of Naive Bayes, Support Vector Machines,
Random Forest, Logistic Regression, J48 Machine learning algorithms and gives \textbf{~95\%+} accuracy in the Train Set. In Test Set, we got \textbf{91.55\%} in Hindi, \textbf{83.44\%} in Malayalam, \textbf{94.20\%} in Punjabi and \textbf{83.44\%} accuracy in Tamil.}
    \item{For Machine Learning portion we have used \textbf{Probabilistic neural network(PNN)} to predict the class.}
    \item{\textbf{Tools} : Python, JAVA, WEKA, MATLAB, XML}
    \item{\textbf{Publication} : Paper published on \textbf{8th meeting of Forum for Information Retrieval Evaluation (FIRE 2016), CEUR-WS.org/Vol-1737, Pages 256-259 }}
      \end{cvitems}
    }
          \cventry
    {Guide : Dr. Partha Pakray, NIT Mizoram, Head of NLP-NITMZ Research Team}
    {QA4FAQ - Question Answering for Frequently Asked Questions, EVALITA-2016}
    {NIT Mizoram, Aizawl}
    {\textbf{Jul. 2016 - Sept. 2016}}
    {
      \begin{cvitems}
        \item{\textbf{Field} : Information Extraction, Information Retrieval, Text Mining}
        \item{Since searching within the Frequently Asked Questions (FAQ) page of a web site is a critical task: customers might feel overloaded by many irrelevant questions and become frustrated due to the difficulty in finding the FAQ suitable for their problems.}
        \item{Developed a search-engine which can effectively retrieve a list of most relevant FAQs and corresponding answers related to the query issued by the user. Used Combinatorics approach for query and by rating each result fetched on a scale like 3,2,1, the most relevant one is shown.}
        \item{Our system can effectively give \textbf{~97\%} relevant search results based on the queries of the user which is much better than any prevalent IR methodologies.}
        \item{\textbf{Tools} : Python, JAVA, Nutch, Apache Tomcat, Italian Stop-word Corpus Building, Combinatorics, Page Rating \& Ranking Algorithms}
        \item{\textbf{Publication} : Paper published on \textbf{3rd Italian Conference on Computational Linguistics (CLiC-it 2016)} }
      \end{cvitems} 
    }
 
  \cventry
    {Guide : Dr. Partha Pakray, NIT Mizoram, Head of NLP-NITMZ Research Team}
    {Information Extraction from Microblogs(\textbf{Twitter}) Posted during Disasters, FIRE-ISI 2016}
    {NIT Mizoram, Aizawl}
    {\textbf{Jul. 2016 - Sept. 2016}}
    {
      \begin{cvitems}
        \item{\textbf{Field} : Social Media Code Mixing, Information Extraction, Information Retrieval}
        \item{Built a system that can deal with the noisy nature of microblogs(\textbf{TWITTER}) which are very small (at most 140 characters) and often written informally, using abbreviations, colloquial terms, etc, and 
effective developed IR methodologies for extracting important information from microblogs posted during disasters.}
        \item{Our system can effectively show the most relevant results related to each topic(Tweets) with high precision Twitter data.}
        \item{\textbf{Tools} : Python, JAVA, Nutch, Apache Tomcat, Word2Vec, Page Ranking Algorithms, Twitter API, JSON, TF-IDF, Skip-gram, Continuous Bag of Words(CBOW)}
      \end{cvitems} 
    }
    \cventry
    {Guide : Prof. (Dr.) Dipankar Das, Jadavpur University}
    {Winter Research Intern, Jadavpur University, Kolkata}
    {Kolkata, India}
    {\textbf{Dec. 2015 - Jan. 2016}}
    {
      \begin{cvitems}
        \item\textbf {Phrase Extraction from English Sentences for Clausal Identification} 
        \item{\textbf{Field} : Information Extraction, Data Mining, Data Structures, Algorithms}
        \item{The system built on the top of Stanford Parser and NLTK can detect various type of Phrases and
 can separate them automatically which can be used to extract Clauses from texts.}
        \item{Developed a recursive algorithm
 based on stack data structure which keep track of the start and end of phrases within Phrases. The task
 is recursively solved to extract the phrases along with their type.}
        \item {\textbf{Tools} : Stanford Parser, NLTK, Python, JAVA}
      \end{cvitems}
    }
\end{cventries}  

